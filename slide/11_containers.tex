\begin{frame}
  \frametitle{主要类别}
  \begin{itemize}
    \item 列表:\texttt{List},数据的一维有序集合
    \item 集合:\texttt{Set},类似列表,但是集合容器内没有重复的元素
    \item 哈希表:\texttt{Map},表示键值对
  \end{itemize}
\end{frame}

\begin{frame}
  \frametitle{x}
\end{frame}

\begin{frame}
  \frametitle{列表List}
  实现了\texttt{List}的主要容器类:
  \begin{itemize}
    \item \texttt{ArrayList}:内部使用数组实现,但是长度可变
    \item \texttt{LinkedList}:内部使用链表结构实现,可以快速的在某一位置删除和插入元素
  \end{itemize}
\end{frame}

\begin{frame}
  \frametitle{List的主要方法}
   \begin{itemize}
    \item \javainline{boolean add(E e)}:添加元素
    \item \javainline{boolean add(int index, E e)}:在指定位置添加元素
    \item \javainline{E set(int index, E e)}:将指位置的元素替换为新元素
    \item \javainline{boolean get(int index)}:取得指位置的元素
    \item \javainline{int indexOf(E e)}:获得元素在容器里的位置
    \item \javainline{boolean remove(int index)}:删除指位置的元素
    \item \javainline{boolean remove(E e)}:删除制定元素
    \item \javainline{boolean removeAll(Collection c)}:从本容器中删除\texttt{c}中指定的所有元素
    \item \javainline{void clear()}:清空本容器
    \item \javainline{void toArray()}:转换为数组
  \end{itemize}

\end{frame}

\begin{frame}
  \frametitle{ArrayList}
\end{frame}

\begin{frame}
  \frametitle{LinkedList}
\end{frame}

\begin{frame}
  \frametitle{哈希表Map}
  实现了\texttt{Map}的主要容器类:
  \begin{itemize}
    \item \texttt{HashMap}:不保证键值对之间的有序性
    \item \texttt{LinkedHashMap}:保证键值对可以保持插入顺序
  \end{itemize}
\end{frame}

\begin{frame}[fragile]
  \frametitle{Map的主要方法}
  \begin{itemize}
    \item \javainline{V get(K key)}:根据\texttt{key}取出一个与之对应的值
    \item \javainline{V put(K key, V value)}:插入一个键值对
    \item \javainline{V remove()}:根据\texttt{key}取出一个元素
    \item \javainline{boolean containsKey(K key)}:移除容器内键为\texttt{key}对应的键值对
    \item \javainline{int size()}:返回容器内键值对的数目
    \item \javainline{boolean isEmpty()}:返回容器是否为空
    \item \javainline{void putAll(map)}:将参数map中的所有键值对插入本容器
    \item \javainline{Set<Map.Entry> entrySet()}:以\texttt{Set}的方式取出容器内所有的键值对,方便遍历
  \end{itemize}
\end{frame}

\begin{frame}[fragile]
  \frametitle{HashMap}

\end{frame}

\begin{frame}[fragile]
  \frametitle{LinkedHashMap}

\end{frame}

\begin{frame}
  \frametitle{Set}
  定义为包含了不重复的元素集合,主要实现类为\texttt{HashSet},主要方法包括:
  \begin{itemize}
    \item \javainline{boolean add(E e)}:
    \item \javainline{boolean contains(E e)}:
    \item \javainline{boolean isEmpty()}:
    \item \javainline{boolean remove(E e)}:
    \item \javainline{boolean removeAll(collection)}:
    \item \javainline{void clear()}:
    \item \javainline{void toArray()}:
  \end{itemize}
\end{frame}

\begin{frame}
  \frametitle{HashSet}
\end{frame}

\begin{frame}
  \frametitle{LinkedList}
\end{frame}

