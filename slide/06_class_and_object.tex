\begin{frame}[fragile]
  \frametitle{类的定义}
  \begin{javacode}
  //定义了类:采用驼峰标记
  public class Person {
    //定义成员变量:驼峰标记法
    String id;
    String name;
    int age;
    
    //定义构造方法
    public Person(String fullName, int age) {
      this.fullName = fullName;
      this.age = age;
    }

    //定义方法:驼峰标记法
    void speak(String something) {
      System.out.println(something);
    }
    
  }
  \end{javacode}
\end{frame}

\begin{frame}[fragile]
  \frametitle{类定义中的要点}
  \begin{itemize}
    \item 使用\javainline{class}关键字定义一个类:
      \begin{javacode}
      class 类名 {
        类体
      } //注意,此处没有“;”
      \end{javacode}
    \item \javainline{class} 后面的花括号括住的内容称为“类体”
    \item 代码风格要求:
      \begin{itemize}
        \item 类名采用大驼峰标记
        \item 包围类体的左花括号写在行尾,前面和类名留一个空格,不要新起一行!
        \item 包围类体的右花括号单独新起一行
      \end{itemize}
      \begin{javacode}
        class Test
        {//错误!左花括号应该在上一行的末尾
        }
        class test //错误!类名使用大驼峰标记
      \end{javacode}
  \end{itemize}
\end{frame}

\begin{frame}[fragile]
  \frametitle{成员变量的声明和初始化}
  \begin{itemize}
    \item 对象的属性也称对象的成员变量(member variable,field)
    \item 成员变量是属于对象的,同一个类实例化出的不同对象虽然有着相同类型的成员变量,但是成员变量的赋值却可以不一样
    \item 也即,成员变量的值是绑定在对象上的,而不是类上的
    \item 成员变量的作用域为整个类体
    \item 定义成员变量的方法和定义变量的方法类似:
      \begin{javacode}
        public class Test {
          int a; // 定义一个成员变量
          int b = 2; // 定义一个成员变量,并赋予默认值(也称为初始化)
          private String c; // 成员变量的前部可以加修饰符
        }
      \end{javacode}
  \end{itemize}
\end{frame}

\begin{frame}[fragile]
  \frametitle{成员变量声明中的代码风格}
  \begin{itemize}
    \item 一行只声明一个成员变量
    \item 如果要为成员变量赋予默认值,等号两边要各有一个空格!
    \item 成员变量的命名采用小驼峰标记法
  \end{itemize}
  \begin{javacode}
    public class Test {
      int a; int c;// 错误!一行只推荐定义一个成员变量
      int b=2; // 错误!等号两边需要有空格
      String MyName; // 错误!请使用小驼峰标记法
      String my_name; // 错误!请使用小驼峰标记法
   }
  \end{javacode}
\end{frame}

\begin{frame}[fragile]
  \frametitle{构造函数}
  \begin{itemize}
    \item 构造函数是定义在类体中,用于初始化对象的一种特殊函数,在创建类的对象时,会调用该类的构造函数
    \item 和类体中的一般方法不同,其函数名必须和类名相同,且无需声明返回值类型,也没有返回值(即不能使用\javainline{return}语句)
    \item 当类体中不定义任何构造函数时,Java会为该类自动添加一个默认构造函数。默认构造函数本质是一个无参的、空函数体的构造函数
    \item 当类体中定义了任何一个构造函数时,Java则不会再为该类田间一个默认构造函数
    \item 在类体中可以使用\javainline{this}关键字指代“当前对象本身”,通过\javainline{this}可以调用本对象的所有方法和属性(不管其控制访问修饰符为何):\javainline{this.[方法名或属性名]}
    \item 使用\javainline{this}可以区分同名变量
  \end{itemize}
\end{frame}

\begin{frame}[fragile]
  \frametitle{构造函数举例}
  \begin{javacode}
    public class Person {
      public String name;
      
      //相当于无参构造函数
      public Person(){
      }
      
      //通过构造函数给成员变量进行初始化赋值
      public Person(String name) {
        //this.name指成员变量name,等号右边的name为构造函数的形参
        this.name = name; 
      }
      
      public String getName() {
        return this.name;
      }
    }  
  \end{javacode}
\end{frame}


\begin{frame}
  \frametitle{定义方法}
  \begin{itemize}
    \item 类体中除了构造函数之外的函数称为“方法”
    \item 如果类的方法签名中定义了返回值类型,方法体中最后需要使用\javainline{return}返回
    \item 如果该方法不返回任何值,应该在方法签名中将函数返回值类型设置为\javainline{void}
    \item 代码风格:
      \begin{itemize}
        \item 方法名采用小驼峰命名法
        \item 同类定义中花括号的使用原则一样,方法体的左花括号写在方法签名行的末尾,右花括号新起一行顶头写
        \item 多个形参之间使用逗号隔开时,逗号后面需要有一个空格:\javainline{int add(int x, int y)}
      \end{itemize}
  \end{itemize}
\end{frame}

\begin{frame}[fragile]
  \frametitle{定义方法举例}
  \begin{javacode}
    public class Person {
      public String name;
      
      public String getName() { //有返回值
        return this.name;
      }
      
      public void printName() { //无返回值
        System.out.println(this.name);
      }
      
      public void setName(String name) {
        this.name = name; //直接使用形参
      }
    }  
  \end{javacode}
\end{frame}

\begin{frame}[fragile]
  \frametitle{对象的创建和使用}
  \begin{itemize}
    \item 对象的创建也称为类的实例化
    \item 使用“\javainline{new 构造函数([args])}”的方式创建对象
    \item 使用\javainline{对象.[方法名或属性名]}的方式访问该对象的方法或属性
  \end{itemize}
  \begin{javacode}
    public class Test {
      public static void main(String[] args) {
        Person p = new Person("Jim");
        System.out.println(p.name); //访问对象的属性
        System.out.println(p.getName()); //调用对象的方法
      }
    }
  \end{javacode}

\end{frame}

\begin{frame}
  \frametitle{静态成员变量和静态方法}
\end{frame}

\begin{frame}
  \frametitle{包}
\end{frame}

\begin{frame}
  \frametitle{访问控制修饰符}
  使用访问控制符来保护对类、变量、方法和构造方法的访问:
  \begin{itemize}
    \item \javainline{default}:默认的访问控制符(即缺省,什么也不写),表示在同一包内可见。使用对象:类、接口、变量、方法
    \item \javainline{private}:在同一类内可见。使用对象:变量、方法。不能修饰类(外部类)
    \item \javainline{protected}:对同一包内的类和所有子类可见。使用对象:变量、方法。不能修饰类(外部类)
    \item \javainline{public}:对所有类可见。使用对象:类、接口、变量、方法
  \end{itemize}
\end{frame}

\begin{frame}[fragile]
  \frametitle{访问控制修饰符的应用举例}
  \begin{javacode}
  public class Person {
    private String name = "x";
    public int age = 21;
  }  
  public class Test() {
     public static void main(String[] args) {
        Person p = new Person();
        p.name; //错误!无法从别地类中访问private修饰的属性和方法
        p.age; //正确
      } 
  }
  \end{javacode}

\end{frame}

\begin{frame}
  \frametitle{外部类与内部类}
\end{frame}

\begin{frame}
  \frametitle{对象的创建和使用}
  \begin{itemize}
    \item 必须使用\texttt{new}关键字
  \end{itemize}
\end{frame}

\begin{frame}
  \frametitle{对象的本质}
  \begin{itemize}
    \item 使用\javainline{new}关键字创建类的一个实例对象,本质是在内存中分配一片内存,并将该内存的指针赋值给新创建的对象变量
    \item 除了基本类型外,所有Java中的对象,本质上都是指针
    \item 所以Java中方法的形参,如果是基本类型,则为传值调用,如果为对象,则为传址调用!
  \end{itemize}
\end{frame}

